\renewcommand*{\familydefault}{\sfdefault}
\renewcommand*{\ttdefault}{txtt}
\newcommand*{\Menu}[1]{\textit{"‘#1"’}}
\newcommand*{\Paket}[1]{\textbf{\mbox{#1}}}
\titleformat{\section}{\normalfont\Huge}{}{0mm}{}
\titleformat{\subsection}%
            {\normalfont\large\bfseries\scshape}{}{0mm}{}
\newcommand{\colbox}[3]{\colorbox{#1}{\textcolor{#2}{#3}}}
\newcommand*{\maincolor}{mittelblau}
\newcommand*{\maintextcolor}{white}
\newcommand*{\SetMainColor}[1]{\renewcommand*{\maincolor}{#1}}
\newcommand*{\SetMainTextColor}[1]{\renewcommand*{\maintextcolor}{#1}}
\newcommand{\Abschnitt}[1]{%
\setlength{\fboxsep}{0.2em}%
\subsection*{\colbox{\maincolor}{\maintextcolor}%
{\parbox{\linewidth-2\fboxsep-2\fboxrule}{#1}}}%
\noindent{}%
}
\newcommand{\Initial}[1]{%
\setcounter{DefaultLines}{3}%
\renewcommand*{\DefaultLoversize}{-0.2}%
\setlength{\fboxsep}{0.4em}%
\lettrine[nindent=-0.1mm,findent=0.7em,lraise=0.3]{%
\colbox{\maincolor}{\maintextcolor}{%
\begin{minipage}{0.9em}%
\centering{}%
#1 %
\end{minipage}%
}%
}{}%
}
\renewcommand*{\labelitemi}{%
\textcolor{\maincolor}{\footnotesize$\blacksquare$}%
}
\newenvironment{Auflistung}
{\begin{itemize}[itemsep=0.0em,leftmargin=*]}
{\end{itemize}}
\newenvironment{Aufzaehlung}{\begin{enumerate}[itemsep=0.0em,leftmargin=*]}{\end{enumerate}}
\newcounter{linkcounter}
\newcommand*{\Link}[2]%
[\arabic{linkcounter}\stepcounter{linkcounter}]%
{~\href{#2}{[#1]}}
\newenvironment{Quellen}[1][\linewidth]
{%
\noindent{}%
\begin{minipage}{#1}
\hfill{}%
\begin{tabular}{r}
\textcolor{dunkelgrau}%
{\normalfont\large\scshape Material de apoyo}\\[-1.9ex]
\textcolor{\maincolor}{\rule{3.7cm}{1.5pt}}\\[-0.8ex]
\end{tabular}
\renewcommand*{\labelenumi}{[\theenumi]}
\begin{small}
\begin{Aufzaehlung}
}
{%
\end{Aufzaehlung}
\end{small}
\renewcommand*{\labelenumi}{\theenumi.}
\end{minipage}
}
\newcommand*{\Quelle}[3]{\item #1{}: \href{#2}{#3}}
\newcommand*{\term}[1]{%
\lstinline[style=StyleListingBasic,%
basicstyle=\ttfamily\bfseries]|#1|%
}
\lstnewenvironment{Befehl}[1][1]
{\lstset{style=StyleCommand,linewidth=#1\linewidth}}
{}
\lstnewenvironment{Listing}[1][1]
{\lstset{style=StyleListing,linewidth=#1\linewidth}}
{}
\newcommand{\Autoreninfo}[3][1.0]{%
\setlength{\fboxsep}{0pt}%
\vskip1em{}%
\noindent\colbox{hellgrau}{black}{%
\parbox{#1\linewidth-2\fboxsep}{%
\setlength{\fboxsep}{4pt}%
\colbox{dunkelgrau}{white}{%
\parbox{\linewidth-2\fboxsep}{{\Large #2}}%
}\\[0.3em]%
\colbox{hellgrau}{black}{%
\parbox{\linewidth-2\fboxsep}{#3}%
}\\[0.3em]%
}%
}%
}
\pagestyle{fancy}
\fancyhead{}
\fancyhead[LE]{Kategorie}
\fancyfoot{}
\fancyfoot[LE,RO]{\textbf{\textsc{\thepage}}}
\fancyfoot[RE,LO]{\textsl{SuperMAG XX/2009}}
\renewcommand*{\headrulewidth}{0.0pt}
\renewcommand*{\footrulewidth}{0.0pt}
\newcommand{\SetFusszeile}[1]{%
\fancyfoot[RE,LO]{\textsl{Llave de Sol, #1}}
\hypersetup{pdftitle={Llave de Sol, #1}}
}
\fancyhead[LE]{\PrintHeader{Kategorie}}
\newcommand{\PrintHeader}[1]{%
  \begin{textblock}{1}(0,0)%
    \begin{sideways}%
      \setlength{\fboxsep}{3pt}
      \colbox{\maincolor}{\maintextcolor}{%
        {\large \textsl{\textbf{#1}}}%
      }%
      \hspace*{1.5cm}%
    \end{sideways}%
  \end{textblock}%
}
\newcommand*{\SetKategorie}[3]{%
\fancyhead{}%
\fancyhead[LE]{\PrintHeader{#1}}%
\SetMainColor{#2}%
\SetMainTextColor{#3}%
}
\newcommand{\fcolbox}[4]{%
\fcolorbox{#1}{#2}{\textcolor{#3}{#4}}%
}
\newcommand{\Bildunterschrift}[3]
{%
\setlength{\fboxsep}{5pt}%
\setlength{\fboxrule}{0.8pt}%
\noindent{}%
\fcolbox{black}{dunkelgrau}{white}{%
\parbox{#1-2\fboxsep-2\fboxrule}{%
\begin{flushleft}%
\vskip-5pt{}%
\textbf{\MakeUppercase{#2}} | #3%
\end{flushleft}%
\vskip-5pt{}%
}%
}%
}
\newcommand{\BildInternal}[4][\linewidth]
{%
\includegraphics[width=#1]{Pictures/#2}\\[-2px]
\Bildunterschrift{#1}{#3}{#4}\\
}
\newcommand{\Bild}[4][\linewidth]
{%
\noindent{}%
\parbox{#1}{\BildInternal[#1]{#2}{#3}{#4}}%
}
\newlength{\VSkipHeight}
\newcommand{\BildAbs}[3]
{%
\begin{textblock}{1}(0,0)%
\Bild[\paperwidth]{#1}{#2}{#3}%
\end{textblock}%
\settoheight{\VSkipHeight}{%
\BildInternal[\paperwidth]{#1}{#2}{#3}%
}%
\addtolength{\VSkipHeight}{-3em}%
\vspace*{\VSkipHeight}%
}
\newcommand{\sbreak}[1][-1em]{\linebreak[4]\\[#1]}
\newcommand{\cbreakfill}{\vfill\columnbreak}
\newcommand{\csbreakfill}{\sbreak\cbreakfill\noindent}
\newcommand{\Einschub}[3]
{%
\begin{wrapfigure}[#1]{#2}{#3}
\vfill
\end{wrapfigure}
}
