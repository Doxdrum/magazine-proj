\BildAbs{BabyGnuLinux.png}{GNU/Linux}{El Gnu (\~nu) y el ping\"uino, las mascotas de Gnu/Linux}

\SetKategorie{Ciencia y Tecnologia}{red}{white}
\SetFusszeile{04/2013}
%% \defaultbibliographystyle{utphys}
%% \defaultbibliography{References}

\begin{article}[2]{Floss}{La ciencia y el software (libre o abierto)}{Oscar Castillo-Felisola}{La ciencia avanza gracias a la tecnolog\'ia, lo que pocos saben es que mucho del software usado es libre o abierto. Acompa\~nanos a conocer esta relaci\'on.}
  
\Initial{E}l desarrollo de la computaci\'on durante el siglo pasado di\'o lugar a que la ciencia avanzara a pasos agigantados, puesto que permit\'io el an\'alisis de datos de manera m\'as eficiente, la creaci\'on de simulaciones num\'ericas, y la aument\'o la velocidad en el proceso creativo.



\subsection{Software libre o abierto}

Generalmente al comprar un computador en una tienda, sobre todo si es en una tienda no especializada, viene instalado un sistema operativo y en algunos casos otros programas para realizar tareas diarias, e.g., navegadores de internet, reproductores de multimedia, etc. 

Estas aplicaciones pueden ser personalizadas (cambiar los colores, el tama\~no, u otro), pero en la mayor\'ia no podemos cambiar su funcionamiento... a\'un si poseemos el conocimiento en programaci\'on para intentarlo. Eso se debe a que las empresas (o personas) que desarrollan el software no permitan el acceso a los c\'odigos fuentes (o sources, en ingl\'es), por lo que se llama {\bf software privativo}.

Existe una tendencia contraria a la creaci\'on de software privativo, en la que se permite al usuario no solo el uso del software, sino tambi\'en al acceso a los c\'odigos fuentes, e incluso es posible manipular las fuentes para cambiar a gusto el comportamiento de los programas. Dicho software es conocido como {\bf software libre o abierto}.

\subsection{Los pioneros}

El movimiento social (y pol\'itico)  en apoyo al software libre se origin\'o con la meta de garantizar al usuario tres libertades:
\begin{itemize}%Auflistung}
\item de usar el software,
\item de estudiarlo y cambiarlo,
\item de redistribuirlo (con o sin modificaciones).
\end{itemize}%Auflistung}
Formalmente el movimiento fue fundado por Richard Stallman en 1983, con el lanzamiento del Proyecto GNU \Link[1]{http://www.gnu.org/gnu/initial-announcement.html}.

No mucho despu'es, en el  a~no 91, un estudiante finland'es, Linus Torvalds, introdujo lo que se conoce hoy como el Kernel de Linux \Link[2]{http://groups.google.com/group/comp.os.minix/msg/b813d52cbc5a044b?dmode=source&pli=1}... que junto al Proyecto GNU di'o origen al sistema operativo GNU/Linux (o simplemente Linux), que no solo puede obtenerse de manera gratuita, sino que existen infinidad de distribuciones Linux. Entre las distribuciones Linux ma's conocidad est'an Debian \Link[3]{http://www.debian.com}, Fedora \Link[4]{http://fedoraproject.org}, Ubuntu \Link[5]{http://www.ubuntu.com}, LinuxMint \Link[6]{http://www.linuxmint.com}, y no puedo dejar de nombrar una variante que se ha masificado increiblemente... Android! \Link[7]{http://www.android.com/}

\subsection{Los grandes titanes}

\Bild{Top500OS.png}{Top500}{Sistema operativo de los Top500 computadores}

Es un hecho que en el universo de los computadores personales, sean escritorios o port'atiles, el mercado est'a dominado por Microsoft (89\%) y Apple (10\%) debido a la distribuci'on de dichos sistemas operativos con la compra inicial del aparato. Esto deja un paque~n'isimo espacio (1\%) para los usuarios del sistema GNU/Linux.

No obstante, en el mundo de los supercomputadores el panorama es totalmente opuesto, pues a la fecha m'as del 90\% de los supercomputadores utilizan GNU/Linux como sistema operativo.

?`D'onde entra la ciencia en este asunto? Los supercomputadores pertenecen en su mayor'ia a universidades y centros de investigaci'on, donde se utilizan como 

\subsection{Industria}

Debido a los avances en ciencia de materiales, la nanotecnolog\'ia revolucionar\'a un sin fin de \'ambitos en la industria en general.

La resistencia de materiales es vital en construcci\'on, sea civil, aeron\'autica o naval. 

Nuevos materiales significar\'ian variaci\'on de costos, mayor durabilidad, mejoras en tiempos de construcci\'on, entre otras.

Nanoestructuras pueden representar una mejora substancial al problema del aislamiento t\'ermico, contribuyendo adem\'as al ahorro energ\'etico.

La inclusi\'on de nanopart\'iculas en mezclas de concreto han resultado en un producto mucho m\'as resistente y durable. El t\'ermino de nano-concreto se utiliza para nombra al concreto hecho de part\'iculas de cemento Portland (hecho de escoria: residuo del proceso de producci\'on de hierro y acero) con di\'ametro de menos de $500 n m$.

\Bild{concrete1.jpg}{Nano-concreto}{Requiere poco material para obtener la misma resistencia y su producci\'on genera un m\'inimo  de emisiones}

En la industria alimenticia, la nanotecnolog\'ia puede ayudar a detectar y/o combatir diferentes tipos de bacterias o microorganismos, contribuyendo al mejor desarrollo y eficiencia de los cultivos.

Las nanopart\'iculas cer\'amicas han mejorado la resistencia al calor y la lisura de equipos caseros, tales como hornos. Adem\'as, pueden usarse nanoestructuras para crean nuevas superficies auto-limpiantes.

Superficies nanom\'etricas se pueden usar como capas ultra-finas de protecci\'on para lentes, as\'i tambi\'en como para anti-reflejo.

En el \'area de la cosmetolog\'ia, podr\'ian realizarse a\~nadiduras a bloqueadores solares para mejorar la duraci\'on de la aplicaci\'on, y la calidad del bloqueado de rayos.

\subsection{Medicina}
\subsubsection*{Diagn\'ostico:}

Nanopart\'iculas magn\'eticas enlazadas con ciertos anti-cuerpos so usadas para atacar mol\'eculas, estructuras o microorganismos  espec\'ificas.

Nanopart\'iculas de oro etiquetadas con segmentos de ADN pueden usarse como detectores de secuencias gen\'eticas en una muestra.

Tecnolog\'ia de nanoporos para el an\'alisis de \'acidos nucleicos convierten directamente cadenas de nucle\'otidos en se\~nales electr\'onicas.

\subsubsection*{Entrega de medicamentos}

Al poder confinar un medicamento dentro de una nanopart\'icula, este puede ser entregado a un grupo espec\'ifico de c\'elulas, reduciendo el consumo de medicamentos por c\'elulas que no lo necesitan. \'Esto permitir\'ia avanzar en una posible %\csbreakfill 
cura del c\'ancer.

Unas nanopart\'iculas como ``bolas de absorci\'on'', pueden usarse para interrumpir la respuesta al\'ergica. \Link[4]{http://www.news.vcu.edu/news/Researchers\_Develop\_Buckyballs\_to\_Fight\_Allergy}

\Bild{nanopista1.jpg}{Nano-inyecci\'on}{Ilustraci\'on de un medicamento conducido a trav\'es de un nanotubo}

\subsubsection*{Ingenier\'ia de tejidos}

La ingenier\'ia de tejidos hace uso de la proliferaci\'on de c\'elulas estimuladas artificialmente al usar ``andamios'' de nanomateriales.

Los huesos pueden regenerarse en torno a ``andamios'' de nanotubos de carbono.

\subsection{Qu\'imica y ambiente}

\subsubsection*{Cat\'alisis}

La cat\'alisis qu\'imica se beneficia de las nanopart\'iculas debido a la gran tasa entre superficie y volumen. Las aplicaciones van desde celdas de combusti\'on hasta dispositivos fotocatal\'iticos.

Las nanopart\'iculas de platino est\'an siendo consideradas para la nueva generaci\'on de catal\'iticos automotriz\Link[5]{http://www.americanelements.com/news\_10\_03\_07.htm}, sin embargo existe controversia debido a posibles riesgos por combusti\'on espont\'anea\Link[6]{http://nanotechweb.org/cws/article/tech/22075} o toxicidad.

\subsubsection*{Filtrado}

Se espera que el tratamiento de aguas residuales sea afectado de manera directa por el desarrollo de materiales foto-qu\'imicos.

Una de las t\'ecnicas de filtrado es el uso de membranas con agujeros de un tama\~no dado.  Membranas nanoporosas, compuestas por nanotubos, pueden filtrar a escalas entre $10-100\;nm$. Un uso directo podr\'ia ser filtrado en di\'alisis renal.

Manomagn\'etos podr\'ian ofrecer formas mucho m\'as eficientes de remover contaminantes de metales pesados de las aguas residuales.



\Autoreninfo{Oscar Castillo-Felisola}{F\'isico Te\'orico, especializado en altas energ\'ias. Actualmente ejerce como Postdoc del Centro Cient'ifico y Tecnol'ogico de Valpara'iso en la UTFSM.}

\begin{Quellen}
  \Quelle{Richard Stallman}{http://www.gnu.org/gnu/initial-announcement.html}{Anuncio del Proyecto GNU.}
  \Quelle{Linus Torvalds}{http://groups.google.com/group/comp.os.minix/msg/b813d52cbc5a044b?dmode=source&pli=1}{email donde se anuncia Linux.}
  \Quelle{SPI Inc.}{http://www.debian.com}{P'agina oficial de Debian Linux.}
  \Quelle{RedHat Inc.}{http://fedoraproject.org/}{P'agina oficial de Fedora Linux.}
  \Quelle{Canonical Ltd.}{http://www.ubuntu.com}{P'agina oficial de Ubuntu linux.}
  \Quelle{Google Cia.}{http://www.android.com/}{P'agina oficial de Android}
\end{Quellen}

%\begin{bibunit}[utphys]
\nocite{Castillo-Felisola:2015nma,Castillo-Felisola:2015ema,Castillo-Felisola:2014xba,Castillo-Felisola:2014iia}
\putbib[References]
%\end{bibunit}

\end{article}
