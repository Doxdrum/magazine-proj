\BildAbs{BabyGnuLinux.png}{GNU/Linux}{El Gnu (\~nu) y el ping\"uino, las mascotas de Gnu/Linux}

\SetKategorie{Ciencia y Tecnologia}{red}{white}
\SetFusszeile{04/2013}

\begin{article}[2]{Floss}{La ciencia y el software (libre o abierto)}{Oscar Castillo-Felisola}{La ciencia avanza gracias a la tecnolog\'ia, lo que pocos saben es que mucho del software usado es libre o abierto. Acompa\~nanos a conocer esta relaci\'on.}
  
\Initial{E}l desarrollo de la computaci\'on durante el siglo pasado di\'o lugar a que la ciencia avanzara a pasos agigantados, puesto que permit\'io el an\'alisis de datos de manera m\'as eficiente, la creaci\'on de simulaciones num\'ericas, y la aument\'o la velocidad en el proceso creativo.

%% %%%%%La idea de este art\'iculo es dar a conocer a grandes rasgos su importancia.\vskip-6em{}
%% {%\vspace*{-3mm}
%% \begin{center}
%%   \begin{tabular}{|>{\columncolor{\maincolor}\color{white}\bfseries}c|c|>{$}r<{$}|}\hline
%%     \rowcolor[gray]{0.8}\color{black} Prefijo & {\bf S\'imbolo} & {\bf (Sub)m\'ultiplo}  \\\hline\hline
%%     %T & Tera & 1000000000000 = 10^{12}\\
%%     G & Giga & 1000000000=10^9\\
%%     M & Mega & 1000000=10^6\\
%%     K & Kilo & 1000=10^3\\
%%     & & 1\\
%%     m & mili & 0,001=10^{-3}\\
%%     $\mu$ & micro & 0,000001=10^{-6}\\
%%     n & nano & 0,000000001=10^{-9}\\\hline
%%     %p & pico & 0,000000000001=10^{-12}%\\
%%     %f & femto & 10^{-15}
%%   \end{tabular}\\[-2px]
%%   \Bildunterschrift{1.203\linewidth}{Escala m\'etrica}{M\'ultiplos y subm\'ultiplos del metro}\\
%% \end{center}
%% }

\Abschnitt{Software libre o abierto}

Generalmente al comprar un computador en una tienda, sobre todo si es en una tienda no especializada, viene instalado un sistema operativo y en algunos casos otros programas para realizar tareas diarias, e.g., navegadores de internet, reproductores de multimedia, etc. 

Estas aplicaciones pueden ser personalizadas (cambiar los colores, el tama\~no, u otro), pero en la mayor\'ia no podemos cambiar su funcionamiento... a\'un si poseemos el conocimiento en programaci\'on para intentarlo. Eso se debe a que las empresas (o personas) que desarrollan el software no permitan el acceso a los c\'odigos fuentes (o sources, en ingl\'es), por lo que se llama {\bf software privativo}.

Existe una tendencia contraria a la creaci\'on de software privativo, en la que se permite al usuario no solo el uso del software, sino tambi\'en al acceso a los c\'odigos fuentes, e incluso es posible manipular las fuentes para cambiar a gusto el comportamiento de los programas. Dicho software es conocido como {\bf software libre o abierto}.


\Abschnitt{Los pioneros}

El movimiento social (y pol\'itico)  en apoyo al software libre se origin\'o con la meta de garantizar al usuario tres libertades:
\begin{Auflistung}
\item de usar el software,
\item de estudiarlo y cambiarlo,
\item de redistribuirlo (con o sin modificaciones).
\end{Auflistung}
Formalmente el movimiento fue fundado por Richard Stallman en 1983, con el lanzamiento del Proyecto GNU \Link[1]{http://www.gnu.org/gnu/initial-announcement.html}.

No mucho despu'es, en el  a~no 91, un estudiante finland'es, Linus Torvalds, introdujo lo que se conoce hoy como el Kernel de Linux \Link[2]{http://groups.google.com/group/comp.os.minix/msg/b813d52cbc5a044b?dmode=source&pli=1}... que junto al Proyecto GNU di'o origen al sistema operativo GNU/Linux (o simplemente Linux), que no solo puede obtenerse de manera gratuita, sino que existen infinidad de distribuciones Linux. Entre las distribuciones Linux ma's conocidad est'an Debian \Link[3]{http://www.debian.com}, Fedora \Link[4]{http://fedoraproject.org}, Ubuntu \Link[5]{http://www.ubuntu.com}, LinuxMint \Link[6]{http://www.linuxmint.com}, y no puedo dejar de nombrar una variante que se ha masificado increiblemente... Android! \Link[7]{http://www.android.com/}





\Abschnitt{Los grandes titanes}

Es un hecho que en el universo de los computadores personales, sean escritorios o port'atiles, el mercado est'a dominado por Microsoft (89\%) y Apple (10\%) debido a la distribuci'on de dichos sistemas operativos con la compra inicial del aparato. Esto deja un paque~n'isimo espacio (1\%) para los usuarios del sistema GNU/Linux.

No obstante, en el mundo de los supercomputadores el panorama es totalmente opuesto, pues a la fecha m'as del 90\% de los supercomputadores utilizan GNU/Linux como sistema operativo.

\Bild{Top500OS.png}{Top500}{Sistema operativo de los Top500 computadores}

?`D'onde entra la ciencia en este asunto? Los supercomputadores pertenecen en su mayor'ia a universidades y centros de investigaci'on, donde se utilizan como 

\Abschnitt{Industria}

Debido a los avances en ciencia de materiales, la nanotecnolog\'ia revolucionar\'a un sin fin de \'ambitos en la industria en general.
%\begin{Auflistung}
%\item 

La resistencia de materiales es vital en construcci\'on, sea civil, aeron\'autica o naval. 
%\item 

Nuevos materiales significar\'ian variaci\'on de costos, mayor durabilidad, mejoras en tiempos de construcci\'on, entre otras.
%\item 

Nanoestructuras pueden representar una mejora substancial al problema del aislamiento t\'ermico, contribuyendo adem\'as al ahorro energ\'etico.
%\item 

La inclusi\'on de nanopart\'iculas en mezclas de concreto han resultado en un producto mucho m\'as resistente y durable. El t\'ermino de nano-concreto se utiliza para nombra al concreto hecho de part\'iculas de cemento Portland (hecho de escoria: residuo del proceso de producci\'on de hierro y acero) con di\'ametro de menos de $500 n m$.

\Bild{concrete1.jpg}{Nano-concreto}{Requiere poco material para obtener la misma resistencia y su producci\'on genera un m\'inimo  de emisiones}
%\item 

En la industria alimenticia, la nanotecnolog\'ia puede ayudar a detectar y/o combatir diferentes tipos de bacterias o microorganismos, contribuyendo al mejor desarrollo y eficiencia de los cultivos.
%\item 

Las nanopart\'iculas cer\'amicas han mejorado la resistencia al calor y la lisura de equipos caseros, tales como hornos. Adem\'as, pueden usarse nanoestructuras para crean nuevas superficies auto-limpiantes.
%\item 

Superficies nanom\'etricas se pueden usar como capas ultra-finas de protecci\'on para lentes, as\'i tambi\'en como para anti-reflejo.
%\item 

En el \'area de la cosmetolog\'ia, podr\'ian realizarse a\~nadiduras a bloqueadores solares para mejorar la duraci\'on de la aplicaci\'on, y la calidad del bloqueado de rayos.
%\end{Auflistung}
%\Bild{concrete0.jpg}{Concreto tradicional}{Requiere gran cantidad de material y su producci\'on genera gran cantidad de emisiones}





\Abschnitt{Medicina}
\subsubsection*{Diagn\'ostico:}

%\begin{Auflistung}
%\item  

Nanopart\'iculas magn\'eticas enlazadas con ciertos anti-cuerpos so usadas para atacar mol\'eculas, estructuras o microorganismos  espec\'ificas.
%\item 

%\Einschub{15}{l}{.5\columnwidth}
Nanopart\'iculas de oro etiquetadas con segmentos de ADN pueden usarse como detectores de secuencias gen\'eticas en una muestra.
%\item 

Tecnolog\'ia de nanoporos para el an\'alisis de \'acidos nucleicos convierten directamente cadenas de nucle\'otidos en se\~nales electr\'onicas.



%\end{Auflistung}

%\Bild[2\columnwidth+1.5\columnsep]{nanopista1.jpg}{Nano-inyecci\'on}{Ilustraci\'on de un medicamento conducido a trav\'es de un nanotubo}


\subsubsection*{Entrega de medicamentos}
%\begin{Auflistung}
%\item 


Al poder confinar un medicamento dentro de una nanopart\'icula, este puede ser entregado a un grupo espec\'ifico de c\'elulas, reduciendo el consumo de medicamentos por c\'elulas que no lo necesitan. \'Esto permitir\'ia avanzar en una posible %\csbreakfill 
cura del c\'ancer.
%\item 
Unas nanopart\'iculas como ``bolas de absorci\'on'', pueden usarse para interrumpir la respuesta al\'ergica. \Link[4]{http://www.news.vcu.edu/news/Researchers\_Develop\_Buckyballs\_to\_Fight\_Allergy}
%\end{Auflistung}
\Bild{nanopista1.jpg}{Nano-inyecci\'on}{Ilustraci\'on de un medicamento conducido a trav\'es de un nanotubo}

\subsubsection*{Ingenier\'ia de tejidos}
%\begin{Auflistung}
%\item 

La ingenier\'ia de tejidos hace uso de la proliferaci\'on de c\'elulas estimuladas artificialmente al usar ``andamios'' de nanomateriales.
%\item 

Los huesos pueden regenerarse en torno a ``andamios'' de nanotubos de carbono.
%\end{Auflistung}



\Abschnitt{Qu\'imica y ambiente}

\subsubsection*{Cat\'alisis}

La cat\'alisis qu\'imica se beneficia de las nanopart\'iculas debido a la gran tasa entre superficie y volumen. Las aplicaciones van desde celdas de combusti\'on hasta dispositivos fotocatal\'iticos.

Las nanopart\'iculas de platino est\'an siendo consideradas para la nueva generaci\'on de catal\'iticos automotriz\Link[5]{http://www.americanelements.com/news\_10\_03\_07.htm}, sin embargo existe controversia debido a posibles riesgos por combusti\'on espont\'anea\Link[6]{http://nanotechweb.org/cws/article/tech/22075} o toxicidad.

\subsubsection*{Filtrado}
%\begin{Auflistung}
%\item 

Se espera que el tratamiento de aguas residuales sea afectado de manera directa por el desarrollo de materiales foto-qu\'imicos.
%\item 


Una de las t\'ecnicas de filtrado es el uso de membranas con agujeros de un tama\~no dado.  Membranas nanoporosas, compuestas por nanotubos, pueden filtrar a escalas entre $10-100\;nm$. Un uso directo podr\'ia ser filtrado en di\'alisis renal.
%\item 

Manomagn\'etos podr\'ian ofrecer formas mucho m\'as eficientes de remover contaminantes de metales pesados de las aguas residuales.
%\end{Auflistung}




%\Abschnitt{}

%\subsubsection*{}
%%\begin{Auflistung}
%%\item 
%%\end{Auflistung}

%\Abschnitt{}

%\subsubsection*{}
%%\begin{Auflistung}
%%\item 
%%\end{Auflistung}



\Autoreninfo{Oscar Castillo-Felisola}{F\'isico Te\'orico, especializado en altas energ\'ias. Actualmente ejerce como Postdoc del Centro Cient'ifico y Tecnol'ogico de Valpara'iso en la UTFSM.}

\begin{Quellen}
  \Quelle{Richard Stallman}{http://www.gnu.org/gnu/initial-announcement.html}{Anuncio del Proyecto GNU.}
  \Quelle{Linus Torvalds}{http://groups.google.com/group/comp.os.minix/msg/b813d52cbc5a044b?dmode=source&pli=1}{email donde se anuncia Linux.}
  \Quelle{SPI Inc.}{http://www.debian.com}{P'agina oficial de Debian Linux.}
  \Quelle{RedHat Inc.}{http://fedoraproject.org/}{P'agina oficial de Fedora Linux.}
  \Quelle{Canonical Ltd.}{http://www.ubuntu.com}{P'agina oficial de Ubuntu linux.}
  \Quelle{Google Cia.}{http://www.android.com/}{P'agina oficial de Android}
\end{Quellen}

\end{article}
