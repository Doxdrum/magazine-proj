\SetKategorie{Computaci\'on}{red!50!blue}{white}
\SetFusszeile{04/2015}
\lead[A different way to present your work]
\begin{article}[2]{beamer}{How to do a Beamer presentation}{Oscar Castillo-Felisola}
  %% Really Basics

  \Initial{I}n order to do a Beamer presentation, one begins as with any LaTeX document, giving a class and beginning and ending the document

  %% \documentclass{beamer}

  %% \begin{document}

  %% \end{document}

  In the preamble (between the class and begin-document) give

  %% \title{...}

  %% \subtitle{...}

  %% \author{...}

  %% \institute{...}

  %% \date{...}

  Now, in the body of the document (between begin-document and end-document)  one may begin to do frames . A frame might contain many slides, but it carries the structure.

  %% \begin{frame}

  %% \end{frame}

  Usually the first frame is the titlepage, so it is done by inserting,

  %% \titlepage

  and the second frame is the outline, so in the second frame one usually adds a frametitle and the content,

  %% \frametitle{Outline}

  %% \tableofcontents

  Before continue the presentation, it is recommended to give it structure in sections and subsections, using the usual

  %% \section{...}

  %% \subsection{...}

  and then one may fill the structure of the document with frames.
  \subsection{Compiling the PDF}

  Now save your file (with extension .tex), say  beamer.tex, and in your terminal type

  %% $ pdflatex beamer.tex

  After compiling, use your favourite PDF viewer to see the result,

  %% $ evince beamer.pdf &

  \subsection{Giving Format}

  As you have notice, the PDF file you have got is quit plain, so the next step is to give a nice format.

  There are a lot of different themes for your presentation, in order to use them, use the command (in the preamble)

  %% \usetheme{...}

  and here is a list of some themes,

  \begin{itemize}
  \item AnnArbor
  \item Berkeley
  \item Berlin
  \item Boadilla
  \item boxes
  \item CambridgeUS
  \item Frankfurt
  \item JuanLesPins
  \item Montpellier
  \item PaloAlto
  \item Warsaw
  \end{itemize}

  If you like the theme structure but not the choice of colours, the command (again in the preamble)

  %% \usecolortheme{...}

  and here is a list of colours,

  \begin{itemize}
  \item beetle
  \item crane
  \item dolphin
  \item dove
  \item lily
  \item orchid
  \item rose
  \item seagull
  \item seahorse
  \item whale
  \item wolverine
  \end{itemize}

  \subsection{Background Picture}

  If you like to add a background picture as a global setting, add the following command line in the preamble

  %% \usebackgroundtemplate{\includegraphics[width=\paperwidth,
  %%     height=\paperheight]{file.jpg}}

  where file.jpg is the name of the picture you’d like to use.

  In instead you prefer to insert the background picture in a single frame, surround the frame with braces and insert the above command line as shown below,

  %% {

  %%   \usebackgroundtemplate{\includegraphics[width=\paperwidth,
  %%       height=\paperheight]{file.jpg}}

  %%   \begin{frame}

  %%   \end{frame}

  %% }

  \subsection{Adding Columns}

  If you want to split a frame in two columns (say to insert text in the left one and a picture in the right one), in the frame include the following command,

  %% \begin{columns}

  %%   \column{.5\textwidth}

  %%   ...content of left column...

  %%   \column{.5\textwidth}

  %%   ... content of right column...

  %% \end{columns}

  \subsection{Sliding a Frame}

  A frame may be divided into several slides by using the command

  %% \pause

  so the compiler generate more than one slide (page of PDF file) for the same frame, each slide shows from the beginning to the corresponding pause.

  When using the itemize environment, one might play with the order items are shown using the command

%% \item<m->

%% \item<-n>

%% \item<m-n>

  if you want the item to appear from the m-th slide, until the n-th slide or from the m-th to the n-th slide.

  \subsection{Using Blocks}

  If one would like to remark some information, there are different blocks structures pre-determined in the beamer class.

  %% \begin{block}{...block title...}

  %%   ...content of the block...

  %% \end{block}

  %% generates a blue-stylish  block of information with custom title.

  %% \begin{alertblock}{...block title...}

  %%   ...content of the block...

  %% \end{alertblock}

  generates a red-stylish block of information with custom title.

  %% \begin{example}[...block subtitle...]

  %%   ...content of the block...

  %% \end{example}

  generates a green-stylish block of information withtitle `Example’ and custom subtitle.

  Additional to these, there are theorem and proof block-environments.
\end{article}
