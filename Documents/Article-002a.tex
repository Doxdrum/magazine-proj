\BildAbs{Nanotube.png}{Nanotubo}{Arreglo hecho de carbono}

\SetKategorie{Ciencia y Tecnologia}{mittelblau}{white}
\SetFusszeile{03/2011}

\begin{article}{Nanostructures}{El advenimiento de la nanotecnolog\'ia}{Oscar Castillo-Felisola}{En los \'ultimos a\~nos la nanotecnolog\'ia se ha convertido en un pilar fundamental del desarrollo, descubramos algunas de sus aplicaciones.}
  
\Initial{S}e dice que el siglo XX fue el siglo de la f\'isica, puesto que se dieron grandes desarrollos en este campo, tales como las teor\'ias especial y general de la relatividad, y la mec\'anica cu\'antica. Sin embargo, el fin de siglo trajo con si la apertura de un \'area extraordinariamente novedosa, que adem\'as de ser atractiva para los f\'isicos te\'oricos, sus aplicaciones han servido para crear una sinergia entre f\'isicos, qu\'imicos, ingenieros e incluso m\'edicos y farmac\'euticos. \'Esta nueva rama de la ciencia es la {\bf nanociencia}.

%%}

\subsection{?`Nano qu\'e?}

Nanoteciencia. Para hacerse una idea sobre lo que es la nanociencia, es necesario recalcar que al ingresar al reino de lo extremadamente peque\~no, el comportamiento de los sistemas es descrito por mec\'anica cu\'antica. As\'i pues, el prefijo {\bf nano} indica que la escala a la se restringe nuestro invitado especial es 9 ordenes de magnitud m\'as peque\~na que la escala de nuestra vida cotidiana (el metro), i.e., $0,000000001$ metros.
\begin{center}
  \begin{tabular}{|>{\columncolor{\maincolor}\color{white}\bfseries}c|c|>{$}r<{$}|}\hline
    \rowcolor[gray]{0.8}\color{black} Pre. & {\bf S\'imb.} & {\bf (Sub)m{\'u}ltiplo}  \\\hline\hline
    %T & Tera & 1000000000000 = 10^{12}\\
    G & Giga & 1000000000=10^9\\
    M & Mega & 1000000=10^6\\
    K & Kilo & 1000=10^3\\
    & & 1\\
    m & mili & 0,001=10^{-3}\\
    $\mu$ & micro & 0,000001=10^{-6}\\
    n & nano & 0,000000001=10^{-9}\\\hline
    %p & pico & 0,000000000001=10^{-12}%\\
    %f & femto & 10^{-15}
  \end{tabular}\\[-2px]
  \Bildunterschrift{1.041\linewidth}{Escala m\'etrica}{M\'ultiplos y subm\'ultiplos del metro}\\
\end{center}

En un nan\'ometro solo hay entre 5 a 10 \'atomos dependiendo del material que se est\'e considerando, por lo que el desarrollo de la nanociencia no fue posible hasta finales del siglo pasado.



\subsection{La historia}

La primera vez que se sugiere la posibilidad de desarrollar tecnolog\'ia en escalas comparables con las escalas at\'omicas, fue el 29 de Diciembre de 1959, cuando  el cient\'ifico Richard Feynman (luego galardonado con el premio Nobel), da una charla  en el encuentro anual de la American Physical  Society, titulada ``There's plenty of room at the bottom'' (Hay much\'isimo espacio all\'a abajo). En esta charla Feynman propone la remota posibilidad de escribir la informaci\'on de la Enciclopedia Brit\'anica en la cabeza de un alfiler\Link[1]{http://www.zyvex.com/nanotech/feynman.html}.

Ciertamente el nombre por el cu\'al hoy se conoce esta \'area no es el que Feynman le di\'o, sino que fue dado muchos a\~nos despu\'es por el profesor Norio Taniguchi, de la Universidad de Tokio.


Solo en la d\'ecada de los 80, con el desarrollo tecnol\'ogico del microscopio de efecto t\'unel o STM (scanning 
tunneling microscope), nace la era de la nanotecnolog\'ia. El STM permite resolver im\'agenes de tama\~no at\'omico.
Poco despu\'es, con la ayuda del STM, se descubren los fulerenos seguido por los nanotubos de carbono (CNT). Otro instrumento inventado para resolver peque\~nas distancias es el microscopio de fuerzas at\'omicas o SFM. Entre otros avances de la \'epoca se tienen  nanocristales semiconductores y  puntos cu\'anticos.

El resto del art\'iculo estar\'a dedicado a describir algunas de las aplicaciones de la nanotecnolog\'ia.% Si bien la nanotecnolog\'ia avanza a pasos agigantados, no es claro para mi, como un no-experto en el tema, el estado actual de cada uno de los t\'opicos mencionados a continuaci\'on.



\subsection{Energ\'ia}
%\subsubsection*{Reducci\'on de consumo energ\'etico}
%\begin{Auflistung}
%\item 
Se puede reducir el consumo energ\'etico al mejorar los sistemas de aislamiento, iluminaci\'on m\'as eficiente o mejorar los sistemas de combusti\'on. Los nanosistemas como LED (Light-Emitting Diodes) o QCA (Quantum Caged Atoms) pueden reducir el consumo el\'ectrico dr\'asticamente.
%\item 

Se puede mejorarla eficiencia de las celdas solares, que hoy d\'ia es de aproximadamente 20\%, al hacer uso de nanoestructuras. Vease por ejemplo \Link[2]{http://news.nationalgeographic.com/news/2005/01/0114\_050114\_solarplastic.html}.
%\item 

Usando catalizadores nanom\'etricos dise\~nados para maximizar el \'area superficial, se puede incrementar la eficacia de los motores de combusti\'on interna.
%\item 

Nanomateriales podr\'ian ser utilizados para crear bater\'ias de  muy larga duraci\'on o de alta tasa de re-cargabilidad, mejorando tambi\'en el problema de desecho de bater\'ias que son extremadamente contaminantes.
%\end{Auflistung}

%\Einschub{7}{l}{1.8cm}


\subsection{Informaci\'on y comunicaci\'on}
\subsubsection*{Almacenamiento}
%\begin{Auflistung}
%\item 
Empresas como {\bf Nantero} y {\bf Hewlett-Packard} son pioneras en el desarrollo de dispositivos de almacenamiento de alta densidad basados en arreglos de nanotubos de carbono. Estos dispositivos son conocidos como \Paket{Nan-RAM} y \Paket{memristor} respectivamente\Link[4]{http://www.youtube.com/watch?v=rvA5r4LtVnc}.
%\end{Auflistung}
\Bild%[\columnwidth+.5\columnsep]
    {memristor.jpg}{Memristor}{Es un arreglo de nano-alambres y nanopuntos que interconectan los alambres, creando transistores en cada intersecci\'on}

\subsubsection*{Dispositivos novedosos}
%\begin{Auflistung}
%\item 

Existen materiales cuya resistencia se var\'ia  al ser sometidos a campos magn\'eticos externos. Si a \'estos se ``implantan'' nanopuntos, el efecto se ve amplificado significativamente, y se conoce como magneto-resistencia-gigante (GMR). \'Este efecto ha conducido a un incremento considerable en la densidad de almacenamiento en discos duros.
%\item 

Otro efecto importante es la magneto-resistencia de efecto tunel (TMR), basado en el tunelaje de electrones a trav\'es de capas adyacentes  de ferro-magnetos dependiendo del spin de los electrones.
%\item 

En 1999 en los Laboratorios para la Tecnolog\'ia Electr\'onica y de la Informaci\'on en Grenoble--Francia, se logr\'o estudiar las propiedades de transistores de $18 nm$, cerca de un d\'ecimo del tama\~no de los transistores comerciales del 2003 y un tercio de los del a\~no 2007.
%\item 

En tecnolog\'ia de comunicaciones cada vez son m\'as los dispositivos el\'ectricos que son reemplazados por dispositivos \'opticos u optoelectr\'onicos, debido a su ancho de banda y capacidad. Entre los ejemplo pueden ser citados los cristales fot\'onicos y los puntos cu\'anticos.
%\item 

Los nanotubos de carbono son conductores el\'ectricos, y pueden ser usados como emisores de campo,%\Einschub{16}{l}{8mm}
 siendo altamente eficientes para pantallas de emisi\'on de campo (FED) las que serian de muy bajo consumo energ\'etico y ultra-alta-definici\'on.
%\item 

Las nanoestructuras proveen un nuevo esquema para fusionar las leyes mec\'anico cu\'anticas con la inform\'atica, creando lo que se conoce como computaci\'on cu\'antica (Quantum Computing).
%\end{Auflistung}


\subsection{Industria}

Debido a los avances en ciencia de materiales, la nanotecnolog\'ia revolucionar\'a un sin fin de \'ambitos en la industria en general.
%\begin{Auflistung}
%\item 

La resistencia de materiales es vital en construcci\'on, sea civil, aeron\'autica o naval. 
%\item 

Nuevos materiales significar\'ian variaci\'on de costos, mayor durabilidad, mejoras en tiempos de construcci\'on, entre otras.
%\item 

Nanoestructuras pueden representar una mejora substancial al problema del aislamiento t\'ermico, contribuyendo adem\'as al ahorro energ\'etico.
%\item 

La inclusi\'on de nanopart\'iculas en mezclas de concreto han resultado en un producto mucho m\'as resistente y durable. El t\'ermino de nano-concreto se utiliza para nombra al concreto hecho de part\'iculas de cemento Portland (hecho de escoria: residuo del proceso de producci\'on de hierro y acero) con di\'ametro de menos de $500 n m$.

\Bild{concrete1.jpg}{Nano-concreto}{Requiere poco material para obtener la misma resistencia y su producci\'on genera un m\'inimo  de emisiones}
%\item 

En la industria alimenticia, la nanotecnolog\'ia puede ayudar a detectar y/o combatir diferentes tipos de bacterias o microorganismos, contribuyendo al mejor desarrollo y eficiencia de los cultivos.
%\item 

Las nanopart\'iculas cer\'amicas han mejorado la resistencia al calor y la lisura de equipos caseros, tales como hornos. Adem\'as, pueden usarse nanoestructuras para crean nuevas superficies auto-limpiantes.
%\item 

Superficies nanom\'etricas se pueden usar como capas ultra-finas de protecci\'on para lentes, as\'i tambi\'en como para anti-reflejo.
%\item 

En el \'area de la cosmetolog\'ia, podr\'ian realizarse a\~nadiduras a bloqueadores solares para mejorar la duraci\'on de la aplicaci\'on, y la calidad del bloqueado de rayos.
%\end{Auflistung}
%\Bild{concrete0.jpg}{Concreto tradicional}{Requiere gran cantidad de material y su producci\'on genera gran cantidad de emisiones}





\subsection{Medicina}
\subsubsection*{Diagn\'ostico:}

%\begin{Auflistung}
%\item  

Nanopart\'iculas magn\'eticas enlazadas con ciertos anti-cuerpos so usadas para atacar mol\'eculas, estructuras o microorganismos  espec\'ificas.
%\item 

%\Einschub{15}{l}{.5\columnwidth}
Nanopart\'iculas de oro etiquetadas con segmentos de ADN pueden usarse como detectores de secuencias gen\'eticas en una muestra.
%\item 

Tecnolog\'ia de nanoporos para el an\'alisis de \'acidos nucleicos convierten directamente cadenas de nucle\'otidos en se\~nales electr\'onicas.



%\end{Auflistung}

%\Bild[2\columnwidth+1.5\columnsep]{nanopista1.jpg}{Nano-inyecci\'on}{Ilustraci\'on de un medicamento conducido a trav\'es de un nanotubo}


\subsubsection*{Entrega de medicamentos}
%\begin{Auflistung}
%\item 


Al poder confinar un medicamento dentro de una nanopart\'icula, este puede ser entregado a un grupo espec\'ifico de c\'elulas, reduciendo el consumo de medicamentos por c\'elulas que no lo necesitan. \'Esto permitir\'ia avanzar en una posible %\csbreakfill 
cura del c\'ancer.
%\item 
Unas nanopart\'iculas como ``bolas de absorci\'on'', pueden usarse para interrumpir la respuesta al\'ergica. \Link[4]{http://www.news.vcu.edu/news/Researchers\_Develop\_Buckyballs\_to\_Fight\_Allergy}
%\end{Auflistung}
\Bild{nanopista1.jpg}{Nano-inyecci\'on}{Ilustraci\'on de un medicamento conducido a trav\'es de un nanotubo}

\subsubsection*{Ingenier\'ia de tejidos}
%\begin{Auflistung}
%\item 

La ingenier\'ia de tejidos hace uso de la proliferaci\'on de c\'elulas estimuladas artificialmente al usar ``andamios'' de nanomateriales.
%\item 

Los huesos pueden regenerarse en torno a ``andamios'' de nanotubos de carbono.
%\end{Auflistung}



\subsection{Qu\'imica y ambiente}

\subsubsection*{Cat\'alisis}

La cat\'alisis qu\'imica se beneficia de las nanopart\'iculas debido a la gran tasa entre superficie y volumen. Las aplicaciones van desde celdas de combusti\'on hasta dispositivos fotocatal\'iticos.

Las nanopart\'iculas de platino est\'an siendo consideradas para la nueva generaci\'on de catal\'iticos automotriz\Link[5]{http://www.americanelements.com/news\_10\_03\_07.htm}, sin embargo existe controversia debido a posibles riesgos por combusti\'on espont\'anea\Link[6]{http://nanotechweb.org/cws/article/tech/22075} o toxicidad.

\subsubsection*{Filtrado}
%\begin{Auflistung}
%\item 

Se espera que el tratamiento de aguas residuales sea afectado de manera directa por el desarrollo de materiales foto-qu\'imicos.
%\item 


Una de las t\'ecnicas de filtrado es el uso de membranas con agujeros de un tama\~no dado. Membranas nanoporosas, compuestas por nanotubos, pueden filtrar a escalas entre $10-100\;nm$. Un uso directo podr\'ia ser filtrado en di\'alisis renal.
%\item 

Manomagn\'etos podr\'ian ofrecer formas mucho m\'as eficientes de remover contaminantes de metales pesados de las aguas residuales.
%\end{Auflistung}




%\subsection{}

%\subsubsection*{}
%%\begin{Auflistung}
%%\item 
%%\end{Auflistung}

%\subsection{}

%\subsubsection*{}
%%\begin{Auflistung}
%%\item 
%%\end{Auflistung}



\Autoreninfo{Oscar Castillo-Felisola}{F\'isico Te\'orico, especializado en altas energ\'ias. Actualmente realiza estudios doctorales en la UTFSM, Valpara\'iso.}

\begin{Quellen}
  \Quelle{Richard Feynman}{http://www.zyvex.com/nanotech/feynman.html}{`There's plenty space at the buttom'}
  \Quelle{Stefan Lovgren}{http://news.nationalgeographic.com/news/2005/01/0114\_050114\_solarplastic.html}{Spray-On Solar-Power Cells Are True Breakthrough}
  \Quelle{Video}{http://www.youtube.com/watch?v=rvA5r4LtVnc}{6-Minute Memristor Guide}
  \Quelle{Abraham, Sathya Achia}{http://www.news.vcu.edu/news/Researchers\_Develop\_Buckyballs\_to\_Fight\_Allergy}{Virginia Commonwealth University Communications and Public Relations. Retrieved 4 November 2010.}
  \Quelle{American Elements}{http://www.americanelements.com/news\_10\_03\_07.htm}{Anuncio de prensa}
  \Quelle{Nanotech web}{http://nanotechweb.org/cws/article/tech/22075}{Technology update from IOP}
\end{Quellen}

\end{article}
